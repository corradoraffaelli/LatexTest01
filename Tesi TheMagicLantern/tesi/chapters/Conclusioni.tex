\chapter{Conclusioni e sviluppi futuri}
\label{chap:risultati}

Al termine dei lavori, possiamo affermare di aver sviluppato un prodotto che convince e getta le basi per l'avviamento di una fase di Produzione definitiva. Possiamo dire ciò anche grazie ai due test svolti, i quali hanno mostrato il potenziale intrattenimento che il gioco riesce a creare, divertendo il pubblico con le sue meccaniche innovative. Se il primo test aveva solo divertito il videogiocatore, il secondo è riuscito anche a fare di più: con le necessarie modifiche, la seconda demo è riuscita a trovare il giusto equilibrio fra divertimento e didatticità, trasmettendo senso di curiosità ai giocatori verso il mondo del pre-cinema (obiettivo primario del progetto), i quali hanno affermato in seguito alla partita, di essere stati incuriositi dalla tematica e di avere persino appreso qualcosa a riguardo.

La meccanica della Lanterna Magica, convince e stupisce i giocatori, ma allo stato attuale vengono mostrate pochissime sue possibili sfaccettature di utilizzo. Difatti, durante la fase di Pre-Production, gli utilizzi ipotizzati (tramite svariati vetrini, sorte di potenziamenti della Lanterna) sono stati tantissimi, ma si è scelto di implementarne solo due per questioni di tempistiche del progetto. Se il progetto dovesse andare avanti, molte delle meccaniche di gioco, solo ipotizzate, potranno essere riprese e riviste meglio, dando al gioco una varietà di elementi davvero vasta. Per esempio, due altri utilizzi di vetrini, che non sono stati inclusi nella demo, sono quelli del ``giocoliere esca'' e dello ``scheletro spaventatore'' (già descritte nel Paragrafo  sulla Lanterna Magica). Mentre il primo attrae l'attenzione dei nemici, facendoli convergere verso il punto dove l'esca è stata proiettata, il secondo li spaventa, facendoli scappare in verso opposto al punto di proiezione. Questi due elementi di gioco, hanno il potenziale per creare molte altre dinamiche e livelli, dando ancor più qualità al prodotto. Inoltre, il secondo tipo di potenziamento (quello dello scheletro) agevolerà l'introduzione nel gioco del ``Fantascopio'', e quindi la relativa fruizione del contenuto informativo. Questo consiste in una evoluzione della Lanterna Magica, ed è stato spesso usato per proiezioni spaventevoli, tramite soggetti quali scheletri e la morte.
A proposito della fruizione dei contenuti Serious, tecniche come l'uso di NPC, schede informative durante i caricamenti, meccaniche di gioco che richiamassero concetti del pre-Cinema, hanno coinvolto in un modo o nell'altro l'utente, trasmettendogli, come già detto, informazioni e curiosità. Anche queste saranno comunque da rivedere e ricalibrare in funzione dell'impostazione del prodotto finale.

A questo punto quindi, ora che il prototipo e le soluzioni proposte sono state validate, il passaggio successivo sarà quello di una nuova sessione di Design che porti alla definizione del prodotto finale che dovrà raggiungere il mercato, mirata anche ad una ristrutturazione dell’idea sulla base di un’attenta analisi delle risorse disponibili, sia umane che economiche. 


\newpage