\chapter{Introduzione}
\label{chap:intro}

Il lavoro di Tesi si pone come obiettivo il design e lo sviluppo di un prototipo di videogioco, della tipologia \textit{Serious Game}, che possa aiutare l’avvicinamento dei ragazzi in età adolescenziale al contesto museale.

È stato svolto presso l’azienda \textit{e-Mentor}, nata a Torino nel 2004 per iniziativa dell’Ing. Manuela Martini, caratterizzata negli anni per lo sviluppo di soluzioni innovative per istituzioni, organizzazioni e aziende in ambito \textit{e-learning}, \textit{web \& mobile design}, \textit{edutainment} e \textit{software engineering} (riferimento).

Il Museo del Cinema di Torino, da tempo pone l’attenzione al problema della poca affluenza di ragazzi in età adolescenziale, cercando soluzioni originali che possano invertire questa tendenza (riferimento).

La collaborazione tra e-Mentor e Museo del Cinema nasce, nel 2013, per provare a porre rimedio a questo problema.
Tra le prime soluzioni proposte c’è quella di sviluppare un videogioco che avvicini i ragazzi a tematiche trattate all’interno del Museo. In seguito a mesi di lavoro e riunioni tra esperti dell’ambito \textit{learning}, artistico e del videogame, viene prodotto un documento di Design, che contiene idee di meccaniche, ambientazioni e scenari di gioco. Le tematiche che si è deciso di trattare sono quelle del \textit{pre-Cinema}, caratterizzato da tecnologie, oggettistica e curiosità per lo più ignoti al pubblico.
Il documento di Design viene presentato al bando \textit{Creative Europe, MEDIA Sub-programme, Support for Concept and Project Development of Video Games}, del Marzo 2014 (riferimento).
L’idea di Design viene accolta positivamente, ma la domanda viene scartata perché il progetto viene considerato poco definito, con alcuni elementi di gameplay non adatti ad essere finanziati e quindi lanciati sul mercato.

Il progetto, ormai denominato \textit{The Magic Lantern}, è stato momentaneamente accantonato, fino al nostro inserimento nel Dicembre del 2014.
Il nostro lavoro è nato con lo scopo di creare un prototipo di gioco funzionante, con cui poter testare idee di design, meccaniche e tecniche di fruizione di contenutistica efficaci.
Le riunioni preliminari si sono quindi focalizzate nel formare un team di lavoro, oltre che stabilire una timeline per organizzare le varie fasi dello sviluppo (riferimento).
L’attività svolta può quindi essere divisa in due grandi momenti:

\begin{itemize}
	\item \textbf{Pre-Production}. Durante la quale ci siamo occupati di una revisione del concept iniziale, aggiornandolo e arricchendolo di elementi interessanti, oltre che della prototipazione delle varie soluzioni proposte così da poterle validare.
	\item \textbf{Production}. Caratterizzata da un raffinamento del design fatto nella fase di pre-production, sviluppo del prototipo, accompagnato dall’art ed audio production e quindi la nascita della First Playable.
\end{itemize}

Per la fase di sviluppo di siamo avvalsi di \textit{Unity3D}, motore di gioco gratuito per la creazione di videogiochi e contenuti interattivi e di \textit{GIT}, come sistema di \textit{versioning}, per la sincronizzazione del lavoro tra i membri del team(riferimento).

Il design del gioco è quindi iniziato con l’idea di sviluppare un prodotto adatto ad un pubblico della fascia di età compresa tra i 12 ed i 16 anni. Le tematiche trattate e le metafore di gioco sarebbero quindi dovute essere coerenti con ragazzi in periodo adolescenziale.
I prodotti sviluppati, in ambito museale, con finalità di intrattenimento, possono porsi in tre differenti momenti in relazione alla visita al museo (riferimento), che vengono definiti come:

\begin{itemize}
	\item \textbf{Pre-Visita}.
	\item \textbf{Visita}.
	\item \textbf{Post-Visita}.
\end{itemize}

Abbiamo quindi inizialmente scartato l’idea di sviluppare un prodotto finalizzato al momento della Visita, perché non coerente con l’obiettivo primario di avvicinamento dei giovani della fascia di età di riferimento. Ci si è quindi focalizzati alla produzione di una soluzione adatta sia ai momenti precedenti che successivi all’ingresso al museo. Questa scelta ha permesso di ideare un prodotto con la funzione di incuriosire i ragazzi nei confronti di tematiche a loro abbastanza sconosciute, così da indurli ad approfondire tali argomenti attraverso una visita al Museo, oltre che adatto a mantenere contatto con i temi eventualmente già apprezzati durante una precedente visita.

Risulta importante specificare come, anche su richiesta del Museo del Cinema, le tematiche non sarebbero dovute essere affrontate in maniera didascalica e didattica, per non rischiare di annoiare o frustrare il giocatore, ma anzi, lo scopo sarebbe dovuto essere esclusivamente quello di incuriosire nei confronti di temi non molto noti.

In seguito all’analisi del contesto ci siamo quindi dedicati al vero design di gioco, scegliendo in primo luogo le principali tipologie di meccaniche da utilizzare (riferimento). Abbiamo valutato la possibilità di usare elementi tipici di videogiochi \textit{platform}, molto intuitivi e familiari, così da facilitare l’immediatezza del prodotto (riferimento). Il gioco è inoltre basato su una importante componente \textit{puzzle}, che ci ha permesso di utilizzare in maniera originale alcune tecnologie del pre-Cinema come metafore di gioco (riferimento).
Secondo quello che si è stabilito durante il design d gioco, questo dovrebbe essere caratterizzato, nel prodotto finito, da ambientazioni caratteristiche, che richiamino gli spettacoli dell’epoca. Poiché, come già specificato, lo scopo del progetto è la creazione di un prototipo atto a verificare l’efficacia di alcune scelte e tecniche usate, la componente artistica si è limitata a quegli elementi che avrebbero potuto influire la comprensibilità del \textit{gameplay}.

Per quando riguarda la componente di gioco \textit{Serious}, si è deciso di fare uso di una meccanica principale, dominante sulle altre, quella della Lanterna Magica, da cui il progetto prende il nome. La Lanterna è considerata l’antenato del proiettore. Generava proiezioni a colori ed in movimento, con un effetto così stupefacente da far credere, al pubblico del ‘600, che le immagini prodotte fossero reali. La meccanica sviluppata è quindi quella di far utilizzare uno strumento capace di cambiare lo scenario con le proiezioni, sulla base dei vetrini disponibili.

Oltre alla meccanica della Lanterna Magica, ne sono state aggiunte altre che rappresentano metafore delle tecnologie del pre-Cinema. Tra queste ci sono le lenti, utilzzate nel gioco per ingrandire e rimpicciolire il personaggio e la Camera Oscura che, al momento dell’interazione, ruota la camera di 180 gradi, così da dare la sensazione che il mondo sia invertito. L’utilizzo di tali meccaniche è stato limitato a piccole porzioni di gioco, in quanto fornire un numero eccessivo di possibilità al giocatore, rischia di confonderlo e sviare l’attenzione verso obiettivi non previsti.

Oltre alle meccaniche di gioco elencate, sono stati inseriti alcuni elementi estetici coerenti con il pre-Cinema, come Lanterne Magiche che generano nemici, o proiezioni caratteristiche.

Per quanto riguarda la fruizione di contenuti, questa è permessa attraverso la lettura di schede informative, ottenibili dal raccoglimento di oggetti nello scenario, dal dialogo con personaggi non giocabili o con il semplice proseguire con l’avventura (riferimento).
I personaggi non giocabili permettono anche di ottenere oggetti collezionabili tramite dei semplici questionari. Questo permette un maggiore interessamento dei giocatori alle schede informative, soprattutto di quelli più portati al completamento dell’avventura in ogni sua componente.

La validazione dello soluzioni sviluppate è avvenuta tramite due sessioni di testing, la prima nel Luglio 2015, presso un oratorio salesiano, la seconda, nel Settembre 2015, presso l’ITIS Majorana di Grugliasco (riferimento).
In seguito all’analisi dei risultati ottenuti dal primo test, abbiamo notato un generale apprezzamento del gameplay e delle meccaniche di gioco, ma un basso interesse verso la tematica del pre-Cinema, sottolineato sia dalla bassa percentuale di apertura delle schede informative, che dalle risposte fornite al questionario anonimo che abbiamo ritenuto utile far compilare a fine sessione (riferimento).
Abbiamo quindi studiato ed implementato delle soluzioni per risolvere questi problemi, tra cui i già citati personaggi non giocabili e le sezioni con lenti e camera oscura. Oltre questo, abbiamo dato la possibilità di osservare contenuti delle schede informative nelle schermate di caricamento e in alcune sezioni particolari dello scenario di gioco (riferimento).
I risultati ottenuti dal secondo test hanno confermato che le soluzioni adottate sono state efficaci. I tester infatti, hanno mostrato un discreto interesse nell’approfondire argomenti presentati tramite meccaniche o solo dal punto di vista estetico, tramite l’apertura e la lettura delle schede informative.

Ora che il prototipo e le soluzioni proposte sono state validate, il passaggio successivo sarà quello di una nuova sessione di Design che porti alla definizione del prodotto finale che dovrà raggiungere il mercato, mirata anche ad una ristrutturazione dell’idea sulla base di un’attenta analisi delle risorse disponibili, sia umane che economiche. 

