% !TEX root = ../thesis.tex
\chapter{The State of the Art}
\label{chap:State of the Art}
What is NFV?

Telecoms networks contain an increasing variety of proprietary hardware appliances.
To launch a new network service often requires yet another appliance and finding the space and power to accommodate these boxes is becoming increasingly difficult, in addition to the complexity of integrating and deploying these appliances in a network.

Moreover, hardware-based appliances rapidly reach end of life: hardware lifecycles are becoming shorter as innovation accelerates, reducing the return on investment of deploying new services and constraining innova on in an increasingly network-centric world.

Network Functions Virtualization (NFV) aims to address these problems by evolving standard IT virtualization technology to consolidate many network equipment types onto industry standard high volume servers, switches and storage. It involves implementig network functions in software that can run on a range of industry standard server hardware, and that can be moved to, or instanciated in, various locations in the network as required, without the need to install new equipment.
Network Functions Virtualization is highly complementary to Software Defined Networking (SDN).
These topics are mutually beneficial but are not dependent on each other. Network functions can be virtualized and deployed without an SDN being required and vice-versa.

accenno ad openflow (utile per dopo quando si parla di ODL/flowrule)